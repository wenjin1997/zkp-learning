% \documentclass[11pt,twoside,a4paper]{article}
% \usepackage{times}

% \usepackage{xeCJK}

% \setmainfont{Times New Roman}

% \setCJKmainfont{Songti SC}
\documentclass[10pt]{ctexart}
% \usepackage[UTF-8]{ctex}
\usepackage{amsmath}
\usepackage{amsthm} % 根据 amsthm 的手册, amsthm 的加载要在 amsmath 之后
\usepackage{amssymb}  %为了能使用\mathbb{H} 
\usepackage{booktabs}
\usepackage{multirow}
\usepackage{tabularx}
\usepackage{xcolor}
\usepackage[colorlinks,linkcolor=blue]{hyperref} % 使用超链接
\usepackage{pdfpages}
\usepackage{geometry}
\geometry{a4paper,scale=0.7}
\usepackage{graphicx} %插入图片的宏包
\usepackage{float} %设置图片浮动位置的宏包
\usepackage{subfigure} %插入多图时用子图显示的宏包
\usepackage{graphicx}

\usepackage{listings}

\lstset{
 columns=fixed,       
 numbers=left,                                        % 在左侧显示行号
 numberstyle=\tiny\color{gray},                       % 设定行号格式
 frame=none,                                          % 不显示背景边框
 backgroundcolor=\color[RGB]{245,245,244},            % 设定背景颜色
 keywordstyle=\color[RGB]{40,40,255},                 % 设定关键字颜色
 numberstyle=\footnotesize\color{darkgray},           
 commentstyle=\it\color[RGB]{0,96,96},                % 设置代码注释的格式
 stringstyle=\rmfamily\slshape\color[RGB]{128,0,0},   % 设置字符串格式
 showstringspaces=false,                              % 不显示字符串中的空格
%language=c++,                                        % 设置语言
}

\newtheorem{definition}{定义}
\newtheorem{lemma}{引理}
\newtheorem{theorem}{定理}
\newtheorem{example}{例}

\usepackage{comment,enumerate,multicol,xspace}

  \newcounter{mnote}
  \setcounter{mnote}{0}
  \newcommand{\mnote}[1]{\addtocounter{mnote}{1}
    \ensuremath{{}^{\bullet\arabic{mnote}}}
    \marginpar{\footnotesize\em\color{red}\ensuremath{\bullet\arabic{mnote}}#1}}
  \let\oldmarginpar\marginpar
    \renewcommand\marginpar[1]{\-\oldmarginpar[\raggedleft\footnotesize #1]%
    {\raggedright\footnotesize #1}}

\title{内积证明}
\author{Jade}
\date{\today}
\begin{document}
\maketitle
\tableofcontents
\section{内积证明}
参考\href{https://dankradfeist.de/ethereum/2021/11/18/inner-product-arguments-mandarin.html}{这篇文章}。

想对这样的结构 $C = \vec{a} \cdot g + \vec{b} \cdot h + (\vec{a} \cdot \vec{b})  q$ 作出承诺。思想是进行折半,折半后的承诺和原来的承诺等价,不断对向量 $\vec{a}, \vec{b}$进行折半,直到最后变成标量,简单计算即可得到证明。

假设 $\vec{a}$ 与 $\vec{b}$ 的向量长度为 $n$,令 $m = \frac{n}{2}$. 记
\begin{equation*}
    z_L = a_mb_0 + a_{m+1}b_1 + \cdots + a_{n-1}b_{m-1} = \vec{a}_R\cdot \vec{b}_L
\end{equation*}
\begin{equation*}
    z_R = a_0b_m + a_{1}b_{m+1} + \cdots + a_{m-1}b_{n-1} = \vec{a}_L\cdot \vec{b}_R
\end{equation*}
从最后得到新的承诺 $C'$ 来看这个构造的过程:
\begin{align*}
    C' & = x C_L + C + x^{-1}C_R \\
    & = x (\vec{a}_R \cdot \vec{g}_L + \vec{b}_L \cdot \vec{h}_R + z_L q) \\
    & \quad + \vec{a}_L \cdot \vec{g}_L + \vec{a}_R \cdot \vec{g}_R + \vec{b}_L \cdot \vec{h}_L + \vec{b}_R \cdot \vec{h}_R + \vec{a} \cdot \vec{b}q \\
    & \quad + x^{-1}(\vec{a}_L \cdot \vec{g}_R +  + \vec{b}_R \cdot \vec{h}_L + z_R q)\\
    & = (x\vec{a}_R + \vec{a}_L)\cdot(\vec{g}_L + x^{-1}\vec{g}_R) \\
    & \quad + (\vec{b}_L + x^{-1}\vec{b}_R)\cdot(\vec{h}_L + x \vec{h}_R) \\
    & \quad + (x z_L + \vec{a} \cdot \vec{b} + x^{-1}z_R)q \\
    & := (x\vec{a}_R + \vec{a}_L) \cdot g' + (\vec{b}_L + x^{-1}\vec{b}_R) \cdot h' + (x z_L + \vec{a} \cdot \vec{b} + x^{-1}z_R)q
\end{align*}
而恰好
\begin{align*}
    (x\vec{a}_R + \vec{a}_L)(\vec{b}_L + x^{-1}\vec{b}_R) & = x \vec{a}_R \vec{b}_L + \vec{a}_L\vec{b}_L + \vec{a}_R\vec{b}_R + x^{-1}\vec{a}_L\vec{b}_R\\
    & = x \vec{a}_R \vec{b}_L + \vec{a} \cdot \vec{b} + x^{-1}\vec{a}_L\vec{b}_R \\
    & = x z_L + \vec{a} \cdot \vec{b} + x^{-1}z_R
\end{align*}
承诺 $C'$ 也满足原来承诺 $C$ 的内积结构。同时 $g'$ 与 $h'$ 相比原来的 $g$ 与 $h$ 已经折半了。 

协议过程:
\begin{enumerate}
    \item Prover计算承诺
        $$
        C_L = \vec{a}_R \cdot \vec{g}_L + \vec{b}_L \cdot \vec{h}_R + z_L q
        $$
        $$
        C_R = \vec{a}_L \cdot \vec{g}_R + \vec{b}_R \cdot \vec{h}_L + z_R q
        $$
    \item Verifier发送挑战 $x \in \mathbb{F}_p$
    \item Prover计算新的向量
        \begin{equation*}
            \vec{a}' = \vec{a}_L + x\vec{a}_R
        \end{equation*}
        \begin{equation*}
            \vec{b}' = \vec{b}_L + x^{-1}\vec{b}_R
        \end{equation*}
    \item Verifier计算新的承诺 $C'$
        \begin{equation*}
            C'  = x C_L + C + x^{-1}C_R
        \end{equation*}
        更新基:
        \begin{equation*}
            \vec{g}' = \vec{g}_L + x^{-1}\vec{g}_R
        \end{equation*}
        \begin{equation*}
            \vec{h}' = \vec{h}_L + x\vec{h}_R
        \end{equation*}
        可以证明,新的承诺$C' = \vec{a}' \cdot \vec{g}' + \vec{b}' \cdot \vec{h}' + \vec{a}' \cdot \vec{b}' q$也是满足内积性质的。
    \item 对$C',\vec{g}',\vec{h}'$重复上述步骤,直到最后向量长度为1,简单计算即可得出等式是否成立。
\end{enumerate}
\end{document}